\section*{Abstract}

This thesis investigates the \textsc{ac}-Stark shift in the ground and excited states of Lithium-6. 
The goal for our experiment is to implement spatially resolved fluorescence imaging of single atoms inside a microscopic dipole trap. To evaluate the feasibility, the trap depth for both the $2s_{1/2}$-ground and the excited $2p_{3/2}$-state is calculated using second order pertubation theory. The shift of resonance frequency of the optical transition for infrared light is predicted and measured using absorption imaging in our crossed dipole trap. The results are used to determine the potential depth of the microtrap. Both theory and experiment find that the Lithium-atom is also trapped in the excited state, but weaker than in the ground state. Therefore single site resolved imaging of individual atoms should be possible.

\section*{Zusammenfassung}

Diese Arbeit untersucht den \textsc{ac}-Stark-Effekt in Grund- und angeregtem Zustand von Lithium-6. Ein Ziel unseres Experimentes ist es, einzelne Atome in einer mikroskopischen Dipolfalle mittels Fluoreszenz-Bildgebung räumlich aufzulösen. Um die entsprechende Realisierbarkeit zu bewerten werden die Fallentiefen für sowohl den $2s_{1/2}$ Grundzustand als auch den angeregten $2p_{3/2}$-Zustand mittels Störungstheorie berechnet. Die Verschiebung der Resonanzfrequenz des entsprechenden optischen Übergangs wird theoretisch berechnet und in unserer optischen Dipolfalle mittels Absorptions-Bildgebung gemessen. Die Resultate werden benutzt um die Potentialtiefe der Mikrofalle zu bestimmen. Es stellt sich heraus, dass der angeregte Zustand in uneserer weit rot verstimmten Dipolfalle ebenfalls gefangen bleibt, allerdings weniger stark als der Grundzustand. Das bedeutet, dass die räumliche Auflösung einzelner Atome möglich sein sollte.