\section*{Abstract}

This thesis investigates the \textsc{ac}-Stark shift in ground and excited states Lithium-6. 
The goal regarding our experiment is to implement spatialy resolved fluorescence-imaging of single atoms inside a microscopic dipole trap. To evaluate the possibilities, the trap depth for both the $2s_{1/2}$-ground and the excited $2p_{3/2}$-state is calculated using the theoretical approach of second order pertubation theory. The shift of the optical transition resonance frequency for our infrared light is predicted and measured using absorbtion imaging in our crossed dipole trap. This enables to find the depth of the used potetials in the microtrap. It is found, that the Lithium-atom is also trapped in the excited state, but weaker than in the ground state which tightens the boundaries for imaging the atoms.

\section*{Zusammenfassung}

Diese Arbeit untersucht den \textsc{ac}-Stark-Effekt in Grund- und angeregtem Zustand von Lithium-6. Ein Ziel unseres Experimentes ist einzelne Atome in einer mikroskopischen Dipolfalle mittels Fluoriszenz-Bildgebung räumlich aufzulösen. Um die entsprechende Möglichkeit zu bewerten werden die Fallentiefen für sowohl den $2s_{1/2}$ Grund- als auch den angeregten $2p_{3/2}$-Zustand mittels quatenmechanischer Störungstheorie berechnet. Die Verschiebung der Resonanzfrequenz des entsprechenden optischen Übergangs wird vorausgesagt und in unserer optischen Dipolfalle mittels Absorbtions-Bildgebung gemessen. Es stellt sich heraus, dass der angeregte Zustand ebenfalls gefangen bleibt, allerdings weniger stark als der Grundzustand, was die Grenzen der Bildgebung in der Falle verengt.