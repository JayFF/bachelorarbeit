\section*{Abstract}

This thesis investigates the \textsc{ac}-Stark shift in ground and excited states of Lithium-6. 
The goal for our experiment is to implement spatially resolved fluorescence imaging of single atoms inside a microscopic dipole trap. To evaluate the feasibility, the trap depth for both the $2s_{1/2}$-ground and the excited $2p_{3/2}$-state is calculated using the theoretical approach of second order pertubation theory. The shift of the optical transition resonance frequency for our infrared light is predicted and measured using absorption imaging in our crossed dipole trap. The results are used to determine the potential depth of the microtrap. It is found theoretically and experimentally, that the Lithium-atom is also trapped in the excited state, but weaker than in the ground state.

\section*{Zusammenfassung}

Diese Arbeit untersucht den \textsc{ac}-Stark-Effekt in Grund- und angeregtem Zustand von Lithium-6. Ein Ziel unseres Experimentes ist einzelne Atome in einer mikroskopischen Dipolfalle mittels Fluoriszenz-Bildgebung räumlich aufzulösen. Um die entsprechende Möglichkeit zu bewerten werden die Fallentiefen für sowohl den $2s_{1/2}$ Grund- als auch den angeregten $2p_{3/2}$-Zustand mittels quatenmechanischer Störungstheorie berechnet. Die Verschiebung der Resonanzfrequenz des entsprechenden optischen Übergangs wird vorausgesagt und in unserer optischen Dipolfalle mittels Absorbtions-Bildgebung gemessen. Die Resultate werden benutzt um die Potentialtiefe der Mikrofalle zu bestimmen. Es stellt sich heraus, dass der angeregte Zustand ebenfalls gefangen bleibt, allerdings weniger stark als der Grundzustand.