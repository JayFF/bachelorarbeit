\chapter{Conclusion and outlook}

The goal of this thesis was to determine the \textsc{ac}-Stark shift in order to evaluate the possibilities of spatially resolved imaging of single atoms in our microtrap setup. It was calculated and measured in the experiment using our crossed dipole trap. The experimental values proved to be in good consistency with the underlying theory. A summary of the results:

For the Lithium-6 ground and excited states ($2s_{1/2}$ and $2p_{3/2}$) the potential of a red detuned dipole trap attractive. The resulting force however is weaker for the excited state. Compared to the ground state, at the used wavelength of 1064 nm, the potential depth for the excited state is  about as 77 \% for $|m_J|=1/2$ and 65 \% for $|m_J|=3/2$ respectively. Inside the microtrap potential this results in a depth of 1 up to 1400 photon recoil energies, depending on the power of the trap beams. Since the kinetic energy directly links to the number of the scattered photons, this also gives a number, how much light can be absorbed and re-emitted, before the atoms are hot enough to escape the traps boundaries. 

Our objective, that is used to catch the fluorescence emission has a high numerical aperture and 10 \% solid angle coverage. The used \textsc{emccd} camera (Andor iXon Ultra 897) is claimed to allow single photon sensitivity. To get a discernible signal we estimate that we should capture at least 10 photons per atom, which would correspond to 100 scattering events. Considering the maximum depth of the trap, site-resolved imaging of single atoms should therefore be possible.

The next step would be to expand the current double well system. Trap stabilization and additional wells would enable new experiments. The first goal would be implementing four individual sites. These could be either arranged in a row or square, being the first block of a 2D optical lattice, but with full control over each individual lattice site. It would be possible to study long-ranged correlations in a linear chain, as well as to analyze the occurrence of many body effects, adding new lattice sites, one by one. 





%In the framework of this experiment the specific question was, whether Lithium-6 atoms in the excited state are also trapping, meaning, that the dipole force acts towards the bigger intensity, as for the ground state. Furthermore the question remains, whether this force is strong enough to hold the atoms in the trap over the imaging process, while absorbing and emitting enough photons to make the fluorescence bright enough to see it in a camera. The first result is, which was also proven in the experiment, that the excited state of the D_2-line is indeed trapping and the dipole force for the same parameters of the used trap is around two thirds of that acting on the ground state.
%
%The interesting question now is whether that is enough to keep the 