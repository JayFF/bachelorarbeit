\chapter{Conclusion and outlook}

The goal of this thesis was to determine the \textsc{ac}-Stark shift in order to evaluate the possibilities of spatially resolved imaging of single atoms in our microtrap setup. It was calculated and measured in the experiment using our crossed dipole trap. The experimental values proved to be in good consistency with the underlying theory. A short short summary of the results:

For the Lithium-6 ground and excited states ($2s_{1/2}$ and $2p_{3/2}$) the potential of our diple trap results in an attractive force. This force is weaker for the excited state. At the used wavelength of 1064 nm the potential depth for the excited state is as much as 77 \% compared to the ground state depth for $|m_J|=1/2$ and 65 \% for $|m_J|=3/2$ respectively. For the used transition this results in 1 to 1400 photon recoil energies, depending on the power of the trap beams. Since the kinetic energy should rise linearly to the number ob the scattered photons \cite{guck das nach Jonathan!}, this gives also the number of photons, scattered before the atom leaves the trap, because of heating. 

Our objective, that is used to catch the fluorescence emission has 10 \% solide angle coverage. The used camera (Andor Stuff, muss nochmal nachschauen) is claimed to allow single photon sensetivity. To get a solid signal we try should try to capture at least 10 photons per atom, which would correspond to 100 scattering events. Considering the maximum depth of the trap there should also be enough headroom.

This means, that it is possible to excite the Lithium-6 atom for the purpose of fluorescence imaging without pushing it out of the trap. Nevertheless, this force is weaker than for the ground states. The trap depth for the excited state therefore is only 65 \% of its ground state value. In the parameters of the microtrap this results in a depth of 1 to 1400 recoil energies in terms of the exciting photons, depending of the laser power. The kinetic energy of the atoms rises linearly \cite{schau nochmal nach Jonathan!} with every recoil, therefore this depths are also valid numbers to determine the maximum amount of scattering before exiting the trap. Our objective, that is used to catch the fluorescence emission has 10 \% solide angle coverage. The used camera (Andor Stuff, muss nochmal nachschauen) claims to allow single photon detection. This means, that 100 scattered photons per atom will be the lower boundary for detection. When ramped up to full power the trap would allow to capture about 14 photons per atom, which should leave enough headroom.



%In the framework of this experiment the specific question was, whether Lithium-6 atoms in the excited state are also trapping, meaning, that the dipole force acts towards the bigger intensity, as for the ground state. Furthermore the question remains, whether this force is strong enough to hold the atoms in the trap over the imaging process, while absorbing and emitting enough photons to make the fluorescence bright enough to see it in a camera. The first result is, which was also proven in the experiment, that the excited state of the D_2-line is indeed trapping and the dipole force for the same parameters of the used trap is around two thirds of that acting on the ground state.
%
%The interesting question now is whether that is enough to keep the 