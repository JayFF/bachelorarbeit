\chapter{Conclusion and outlook}

The goal of this thesis was to determine the \textsc{ac}-Stark shift in order to evaluate the possibilities of spatially resolved imaging of single atoms in our microtrap setup. It was calculated and measured in the experiment using our crossed dipole trap. The experimental values proved to be in good consistency with the underlying theory. The main result however is the insight that both states of the D2-line feel an attractive force in the trap potential. This means, that it is possible to excite the Lithium-6 atom for the purpose of fluorescence imaging without pushing it out of the trap. Nevertheless, this force is weaker than for the ground states. The trap depth for the excited state therefore is only 67 \% of its ground state value. In the parameters of the microtrap this results in a depth of 1 to 1400 recoil energies in terms of the exciting photons, depending of the laser power. This amount of scattered photons has led to single atom detections in comparable experiments \cite{schmiedmayer}.

The next step is to implement a setup to realize the imaging method. The current plan is to use a resonant laser beam, that normally is used for our absorption imaging and retro-reflect it to cancel out the net radiation pressure. The resulting light will be caught by a high numerical aperture objective. 


In the framework of this experiment the specific question was, whether Lithium-6 atoms in the excited state are also trapping, meaning, that the dipole force acts towards the bigger intensity, as for the ground state. Furthermore the question remains, whether this force is strong enough to hold the atoms in the trap over the imaging process, while absorbing and emitting enough photons to make the fluorescence bright enough to see it in a camera. The first result is, which was also proven in the experiment, that the excited state of the D_2-line is indeed trapping and the dipole force for the same parameters of the used trap is around two thirds of that acting on the ground state.

The interesting question now is whether that is enough to keep the 