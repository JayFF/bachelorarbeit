\chapter{The Experiment}


The key feature of the experiment is the possibility to prepare few atoms in a microscopic dipole trap with high control over their number. A precise description of the experiment is found in \cite{friedhelm}. A briefer summary of this is given in this paragraph.

The experiments are carried out in the so called microtrap, that for the lowest occupied levels can be approximated by a one-dimensional harmonic oscillator. To fill this potential down to the ground state a big reservoid of cold atoms has to be prepared before activating this main stage of the experiment.

After vaporizing Lithium in an oven, the first component of the experiment is the Zeeman slower. When leaving the oven, the atoms are very hot and propagate at velocities, that are too large to trap them in the \textsc{mot} or dipole trap. The Zeeman-slower reduces this speed using radiation pressure while adapting for the the different doppler shifts, exploiting the Zeeman-effect. The slower is essentially a large tube behind the oven shutter. It is surroundet by coils, that provide a magnetic field of different values at every point inside the apperatus. A strong laserbeam is pointing along the slower. When the beam is in resonance with an atomic transition, the photons get absorbed and the atoms are pushed in the direction of the laser. Because the spontanious emission is directed in random directions, this leads to slowing of the cloud. The atoms however, due to the Doppler-effect see the light blue detuned when flying at high velocities. If the frequency of the laser-beam is adapted to match the resonance frequency for the respective speed, the atoms will be slowed down, but therefore will no longer absorb photons of the same frequency. Therefore different magnetic fields shifts the atomic levels to match the frequncy of the laser everywhere along the legth of the tube and achieve enough cooling to trap the atoms in the \textsc{mot}. 
\begin{figure}[h]
\centering
\begin{subfigure}[b]{0.8\textwidth}
                \includegraphics[width=\textwidth]{exsetup}
\end{subfigure}
\caption{Model of the experiments core-components. On the left the octagon-shaped vacuum chamber is visible, in which the experiment is carried out. Trough the viewports lasers for the \textsc{mot} and dipole-trap enter the chamber, (see figure \ref{scheme}). It is surrounded by coils providing magnetic field for several stages of the process. On the right one can see the oven and the Zeeman-slower connecting both parts.}
\label{experiment}
\end{figure}
It consists of six counterpropagating near-resonsant laser-beams. The usage of many retro-reflected beams in different directions makes it possible not only to force the atoms in a certain direction like in the Zeeman-slower, but to also affect all of them with a high absolute speed value and push them into the opposite way, in this manner reducing the overall average-speed and therefore cooling the gas. However, since the force of the laserbeams alone is only velocity-dependent, slow particles would exit the center of the crossed beams over time. For this reason a magneto-optical trap uses a magnetic quadrupole-field, that has zero strength in the middle and increases when moving further away from the center. The Zeeman-effect shifts the level-distace for the outmoving atoms towards the frequency of the laser, which will then apply a force, dependent on the spatial position, enabeling not only cooling but trapping and compression of the gas-sample. The natural linewith of the used transition limits the cooling of the \textsc{mot} to a temperature of around 140$\unit{\mu K}$. The system used in this experiment can store around 10⁸ atoms. 

The temperature however has to be reduced much further to match the requirements of the experiment later on. The next step on this ladder is the crossed dipole trap. 
%\begin{figure}[h]
%\centering
%\begin{subfigure}[b]{0.8\textwidth}
%                \includegraphics[width=\textwidth]{dipolefoto}
%\end{subfigure}
%\caption{Absorbtion image of the crossed dipole trap \cite{lompe}.}
%\label{experiment}
%\end{figure}
It uses the laserbeam of a 200 \textsc{w} Ytterbium fiber-laser (\textsc{ipg ylr-200-lp}) that is far red-detuned from the atomic transitions at 1070 nm. The beams focus lies within the \textsc{mot}, with a waist of approximately 40 \mu m \cite{lompe}. After leaving the vacuum chamber, a mirror system guides the laser beam back at a different angle, forming the crossed trap shape. To trap the most atoms possible the laser is ramped-up to full power, generating a trap depth of more than 3 mK. To further cool the sample the power is slowly ramped down, so the hottest atoms escape the potential and the rest of the cloud thermalises at a lower temperature (evaporative cooling). A small, tightly focussed infrared laser-beam at 1064 nm, that intersects the crossed dipole-trap then formes the microtrap (\textsc{mephisto S} from Innolight). The small dimensions of the trap result in high spacing between the allowed harmonic vibrational levels. To control the atom number in the microtrap, a magnetic field gradient tilts the dipole-potential. That leads to the escape of all atoms above a certain level. Uing this technique enables to control the number of atoms between 0 and 10 with high preperation fidelity. To see the atoms, the microtrap is shut down and the remaining atoms are again trapped in a smaller \textsc{mot}. The resulting fluorescence is caught by an objective and projected at the sensor of a \textsc{ccd}-camera. 
\begin{figure}[h]
\centering
\begin{subfigure}[b]{0.8\textwidth}
                \includegraphics[width=\textwidth]{scheme}
\end{subfigure}
\caption{Scheme of the laser set-up around the vacuum chamber \cite{lompe}. }
\label{scheme}
\end{figure}
\section{Imaging}

To understand how that has been done, the different methods of imaging used in the setup are described briefly.

\subsection{Absorbtion imaging}

Absorbtion imaging is a common method to image clouds of atoms\cite{ketterle}. A resonant beam is pointed on a camera, through the sample, and partly gets absorbed. Therefore a shadow is visible on the sensor, that shows the density distribution along the cloud. By fitting a gaussian curve on that distribution the atom-number can be estimated.

In this setup we use a tunable, grating stabilized diode laser (Toptica \textsc{dl}-100). To get usable noise-free data three images has to be taken. The first is taken, when the actual imaging is done, with the atom cloud in the beam path. The next is taken when the atom cloud has cleared, with pure laser-light to account for noise of the light source. The last one is taken in the dark to account for disturbance of the ambient light. When both corrections are subtracted one gets a clear image of the atomic cloud.

This technique is very efficient in imaging clouds of atoms, because the measured time-period is short. Therefore also time-of-flight measurements can be done, when taking pictures after waiting a certain interval, that enables not only to see the form of the cloud and estimate the atom numbers, but also measure the particles velocities and therefore their momentum.

However, this method is no suitable for resolving single atoms, because a certain density is necessary to make an estimate, using the gaussian fit. For this one has to rely on the more basic fluorescence imaging.

\subsection{fluorescence imaging}

The method of fluorescence imaging is very basic\cite{timo}. Resonant light excites the atoms, that spontaneously re-emit photons, that can be caught on camera. In a large \textsc{mot} this method is canonical, because the trapping beams are in resonance and the resulting fluorescence can even be seen with the naked eye. When imaging smaller amounts of particles the process is less trivial, because spontaneous emission is directed randomly and only a small portion of the emitted light is caught on camera. That means a larger amount of emitted and therefore absorbed photons is necessary to get a visible result. But in contrast to absorption imaging, it is possible in our setup to even image single atoms. After they are released from the microtrap a smaller \textsc{mot} is activated, that can hold the atoms long enough to catch enough light on the sensor. This however only enables to count the number of the atoms. By knowing the parameters used for preparation it is possible to identify the quantum state of the system before release. Yet in a multi well potential or optical lattice it can be also very useful to resolve the atoms spatially in their respective lattice sites. When exciting the atoms while still keeping the trap lasers activated the question is now, whether the atoms can be stored long enough in their potential to emit enough light to see them.

