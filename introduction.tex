\chapter{Introduction}

Experimenting with cold atoms has become a major part of modern physical research and has let to new insights in the fields of quantum simulation and quantum information, shedding light also on other fields in physics from high energy to condensed matter. A vital set of techniques, enabeling such experiments is based on trapping, storing and cooling atoms with conservative forces, induced by the interaction of the dipole-moment with far detuned laser-light. A commonly used concept for preparing the investigated system is the optical lattice. Using interfering beams, that form a standing wave, a landscape of periodic potentials is created that can be used to resemble the lattice-structure in a solid cristal enabeling the simulation of many effects in many-body physics such as supercondctivity, superfuiditiy or Mott inslation. To actually perform experiments in such a potential-structure it is of vital importance to information about the atoms behaviour while interacting in the lattice. This however is not a trivial problem, since the only possibility to get an insight into the state of the system is by driving atomic transitions and either collecting photons, showing the position of the atoms or using the absorbtion to get an inverse image of the cloud. These measurements are mostly done short periods after deactivating the trap-beams. While absorbtion imaging can in principle be done within the trap or lattice, it is not suitable to resolve low numbers of particles. Flourescence imaging on the other hand can be accurate enough to count single atoms but so far few approaches have been implemented, that are able to perform this technique inside the trap. A part of the issue is, that an atom that is excited for using its spontanious emission has a different induced dipole moment and reacts differently to the used lasers than the groundstate, that is initially trapped in the experiment. Depending on the atomic species, the trapping forces could be higher, lower or even repulsive in an excited state, rendering detection of enough flourescence impossible without messing up the distribution inside the lattice or dissolving the cloud as a whole. On goal in our experiment is to use the flourescence of the D2-line in Lithium-6 to image single sites in a multiwell-potential in an infrared dipole-trap. To achieve this it has to be investigated whether the excited state is still trapping and whether the trapping force is stron enough to hold the atoms long enough to collect enough spontanious radiation.
\section{Outline}
In this thesis the \textsc{ac}-Stark shift of the ground and the excited state of the atomic transition are calculated and tested in experiment. This enables the prediction of trap depths for each state and allows conclusions regarding the seeked flourescence imaging inside dipole-traps. The thesis starts by giving a general overview about the experiment with its features and goals. The next part shortly describes the theoretic foundation of the atomic structure followed by a detailed analysis of the interaction of particles with electromagnetic waves, leading to the formulas later used to calculate the \textsc{ac}-Stark shift for the $2s_{1/2}$ and $2p_{3/2}$ states 