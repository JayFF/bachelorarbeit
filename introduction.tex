\chapter{Introduction}

Many-body effects are of great interest for modern physics and many aspects of this field have vital importance not only for fundamental research, but for effects encountered and used in daily life. The behaviour of electrons in a crystal for example is such a problem, being the basis for many properties of condensed matter, such as the differences between metal, insulators and semiconductors, being of fundamental concern for all kinds of electronical applications. Because of the long range Coulomb force and the high density of the interacting electrons the underlying physics can only be resolved when considering many-body systems, that sadly can not be calculated easily but require complex and time-consuming simulations, when treated numericaly. Studying the real systems in contrast, i.e. solid crystals or other materials, is similarily complicated, because solid materials limit the control over the paramters, used in the describing models, ranging from the temperature of the system to the form and depth of the atomic potentials. A promissing field of research, that aimes to overcoming all these problems, are ultracold gases. By exploiting their interaction with laser-light one can cool, trap and store atoms, using devices like magneto optical traps or optical dipole traps\cite{metcalf}. The last approach allows to create potential landscapes, that emulate structures in solid crystals, which allows to shape the Hamiltonian of such systems to resemble those of theoretical models for condensed matter physics, as well as tuning the relevant parameters with good precision. This approach is named Quantum simulation.

Our experiment uses a dipole trap that forms a so called double-well potential for fermionic Lithium-6 atoms. Using two independent laser-beams and expoiting the effect of feshbach resonances this system emulates the builing block for the Hamiltonian of a simple but very important model in condensed matter physics, the Hubbard model \cite{doublewell}.

An important issue in this sense and for every other experiment with cold gases is to determine the state of the system that is simulated. As the atomic clouds have to be isolated in vaccum chambers, the only means to do that rely on the interaction with laser-light. The first approach is absorbtion imaging\cite{ketterle}. Laser-light, that is in resonance with an atomic transition is absorbed by the atom cloud and the shadow can be caught on a camera. When varying the time of flight after deactivating the traps, the density and momentum distribution can be measured. However, to resolve low numbers of particles in the trap this method is not suitable, since a certain density of atoms is nessesairy. For our experiment that enables us to prepare single atoms another approach has to be taken. The flourescence of the excited atoms is caught, so far enabeling us to count the exact number of atoms in the experiment, loading them into a small magneto optical trap. The next step however is to use flourescence imaging to resolve the spacial position in the trap, making it possible to determine the quantum state of the system more precisely. The issues lie in the nature of the dipole-trap that is used to create the potential. The interaction of the electric field component of laser-light with the atomic dipole moment leads to an energy-shift of the atomic levels, the \textsc{ac}-Stark effect. This level shift and the resulting potential are dependent of the atomic state that is changing when exciting the atom to produce flourescence. Therefore the possibilities of trapping the particles in this process rely on the properties of the involved levels. 

In this thesis the \textsc{ac}-Stark shift of the ground and the excited state of the D2-line in Lithium-6 are calculated and tested in experiment. This enables the prediction of trap depths for each state and allows conclusions regarding the flourescence imaging inside dipole-traps and optical lattices. The thesis starts by giving a general overview about the experiment with its features and goals. The next part shortly describes the theoretic foundation of the atomic structure followed by a detailed analysis of the interaction of particles with electromagnetic waves, leading to the formulas later used to calculate the \textsc{ac}-Stark shift for the $2s_{1/2}$ and $2p_{3/2}$ states. After that the results of calculation and measurement of the predicted values are presented.

%Experimenting with cold atoms has become a major part of modern physical research and has let to new insights in the fields of quantum simulation and quantum information, shedding light also on other fields in physics from high energy to condensed matter. A vital set of techniques, enabeling such experiments is based on trapping, storing and cooling atoms with conservative forces, induced by the interaction of the dipole-moment with far detuned laser-light. A commonly used concept for preparing the investigated system is the optical lattice. Using interfering beams, that form a standing wave, a landscape of periodic potentials is created that can be used to resemble the lattice-structure in a solid cristal enabeling the simulation of many effects in many-body physics such as supercondctivity, superfuiditiy or Mott insulation. To actually perform experiments in such a potential-structure it is of vital importance to get information about the atoms behaviour while interacting in the lattice. This however is not a trivial problem, since the only possibility to get an insight into the state of the system is by driving atomic transitions and either collecting photons, showing the position of the atoms or using the absorbtion to get an inverse image of the cloud. These measurements are mostly done short periods after deactivating the trap-beams. While absorbtion imaging can in principle be done within the trap or lattice, it is not suitable to resolve low numbers of particles. Flourescence imaging on the other hand can be accurate enough to count single atoms but so far few approaches have been implemented, that are able to perform this technique inside the trap. A part of the issue is, that an atom that is excited for using its spontanious emission has a different induced dipole moment and reacts differently to the used lasers than the groundstate, that is initially trapped in the experiment. Depending on the atomic species, the trapping forces could be higher, lower or even repulsive in an excited state, rendering detection of enough flourescence impossible without messing up the distribution inside the lattice or dissolving the cloud as a whole. On goal in our experiment is to use the flourescence of the D2-line in Lithium-6 to image single sites in a multiwell-potential in an infrared dipole-trap. To achieve this it has to be investigated whether the excited state is still trapping and whether the trapping force is stron enough to hold the atoms long enough to collect enough spontanious radiation.
%\section{Outline}
%In this thesis the \textsc{ac}-Stark shift of the ground and the excited state of the atomic transition are calculated and tested in experiment. This enables the prediction of trap depths for each state and allows conclusions regarding the seeked flourescence imaging inside dipole-traps. The thesis starts by giving a general overview about the experiment with its features and goals. The next part shortly describes the theoretic foundation of the atomic structure followed by a detailed analysis of the interaction of particles with electromagnetic waves, leading to the formulas later used to calculate the \textsc{ac}-Stark shift for the $2s_{1/2}$ and $2p_{3/2}$ states 