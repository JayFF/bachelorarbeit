\chapter{Introduction}

Experimenting with cold atoms has become a major part of modern physical research and has let to new insights in the fields of quantum simulation and quantum information, shedding light also on other fields in physics from high energy to condensed matter. A vital set of techniques, enabeling such experiments is based on trapping, storing and cooling atoms with conservative forces, induced by the interaction of the dipole-moment with far detuned laser-light. A commonly used concept for preparing the investigated system is the optical lattice. Using interfering beams, that form a standing wave, a landscape of periodic potentials is created that for example is used to represent the lattice-structure in a solid cristal. To actually perform experiments in such a potential-structure it is of vital importance to get as much information about the atoms behaviour as possible while interacting in the lattice. This however is not a trivial problem, since the only possibility to get an insight into the state of the system is by driving atomic transitions and collecting photons, showing the position of the atoms. 